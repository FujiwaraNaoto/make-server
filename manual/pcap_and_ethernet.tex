% encoded in UTF-8
\documentclass[a4paper]{article}%日本語パッケージに必要らしい
\usepackage[whole]{bxcjkjatype}%日本語パッケージに必要らしい
%\documentclass[a4j]{jarticle}
\usepackage{comment}
\usepackage{listings,jvlisting} %日本語のコメントアウトをする場合jvlisting(もしくはjlisting)が必要
%ここからソースコードの表示に関する設定
\setcounter{secnumdepth}{4}%4階層まで根付の深さの指定
\begin{document}
\part{Network}
\section{pcap}

まず pcap.hを使用する.pcap.hをインストールするにはターミナル上で

sudo apt-get install libpcap-dev 

と入力する. pcap.hを使用したソースコードをコンパイルする際は

gcc -lpcap -Wall ソースファイル.c -o 実行ファイル

と入力する. できた実行ファイルは管理者権限下で実行する必要があり, 実行の際には

sudo ./実行ファイル名

とターミナルに入力する.
\begin{lstlisting}[caption=pcap\_loopに使用するcallback関数の仕様,label=fuga]
void callback
(u_char *args,const struct pcap_pkthdr *cap_header, const u_char *packet)
\end{lstlisting}

pcap\_pkthdr構造体は以下のようになっている.
\begin{lstlisting}[caption=pcap\_pkthdr構造体,label=struct pcap_pkthdr]
struct pcap_pkthdr
{
  struct timeval ts;
  buf_u_int32 caplen;
  buf_uint32 len;
}
\end{lstlisting}




\section{Ethernet}
これは net/inet.hで定義されている.
\begin{lstlisting}[caption=ethernet\_header構造体,label=struct ethernet]
struct ether_header
{
  uint8_t  ether_dhost[ETH_ALEN]; /* destination eth addr */
  uint8_t  ether_shost[ETH_ALEN]; /* source ether addr */
  uint16_t ether_type;              /* packet type ID field */
} __attribute__ ((__packed__));
\end{lstlisting}
\ref{struct ethernet} 中の \_\_attribute\_\_ ((\_\_packed\_\_))は 構造体の隙間を詰めるものであり,本質ではない.
ether\_typeは以下のように定義される. \\
ETHERTYPE\_PUP  \\
ETHERTYPE\_IP: IPv4 \\
ETHERTYPE\_ARP:Address Resolution Protocol \\
ETHERTYPE\_AT:Apple Talk Protocol\\
ETHERTYPE\_AARP:Apple Talk ARP \\
ETHERTYPE\_REVARP:Reverse ARP \\
ETHERTYPE\_VLAN: VLAN tagging \\
ETHERTYPE\_IPX:IPX\\
ETHERTYPE\_IPv6:IPv6 \\
ETHERTYPE\_LOOPBACK:loopback\\
なお,実装の際には短整数をネットワークバイトオーダーからホストバイトオーダーへと変換させる関数htonsを使用して
 ntohs(ether\_type)とする必要がある. 次のような実装を考えることができる.
\begin{lstlisting}[caption=hoge,label=hoge]
    switch(ntohs(eth_hdr->ether_type)){
        case ETHERTYPE_PUP:
            printf("(Xerox PUP)\n");
            break;
        
        case ETHERTYPE_IP:
            printf("(IP)\n");
            break;
        
        case ETHERTYPE_ARP:
            printf("Address resolution\n");
            break;

        case ETHERTYPE_AT:
            printf("Appe Talk protocol\n");
            break;
            
        case ETHERTYPE_AARP:
            printf("Appe Talk ARRP\n");
            break;
            
        case ETHERTYPE_REVARP:
            printf("Reverse ARP\n");
            break; 
            
        case ETHERTYPE_VLAN:
            printf("VLAN tagging\n");
            break;
            
        case ETHERTYPE_IPX:
            printf("IPX\n");
            break;
            
        case ETHERTYPE_IPV6:
            printf("IPv6\n");
            break;
            
        case ETHERTYPE_LOOPBACK:
          printf("loop back\n");
          break;
          
        default:
            printf("unknown");
            break;

\end{lstlisting}

配列ether\_dhostとether\_shostは次のように書式指定子を用いて16進数に書き換えることが必要.
\begin{lstlisting}[caption=hoge,label=hoge]
dmac[18]={0}:
u_char *hwaddr = ether_dhost;
snprintf(dmac,sizeof(dmac),"%02x:%02x:%02x:%02x:%02x:%02x",
                     hwaddr[0],hwaddr[1],hwaddr[2],
                     hwaddr[3],hwaddr[4],hwaddr[5]);
\end{lstlisting}

以上をまとめるとpcap\_loopに次のようなcallback関数を作ることができる.
\begin{lstlisting}[caption=hoge, label=hoge]


char * convmac_tostr(u_char *hwaddr,char *mac,size_t size){
  snprintf(mac,size,"%02x:%02x:%02x:%02x:%02x:%02x",
                     hwaddr[0],hwaddr[1],hwaddr[2],
                     hwaddr[3],hwaddr[4],hwaddr[5]);
  return mac;
}

void start_pktfunc( u_char *user,      
                    const struct pcap_pkthdr *h , 
                    const u_char *p        
                     ){
  char dmac[18] = {0};
  char smac[18] = {0};
  struct ether_header *eth_hdr = (struct ether_header *)p;

  printf("ether header---------\n");
  printf("dest mac %s\n",convmac_tostr(eth_hdr->ether_dhost,dmac,sizeof(dmac)));
  printf("src mac %s\n",convmac_tostr(eth_hdr->ether_shost,smac,sizeof(smac)));
  printf("ether type %x\n\n",ntohs(eth_hdr->ether_type));
    switch(ntohs(eth_hdr->ether_type)){


    switch(ntohs(eth_hdr->ether_type)){
        case ETHERTYPE_PUP:
            printf("(Xerox PUP)\n");
            break;
        
        case ETHERTYPE_IP:
            printf("(IP)\n");
            break;
        
        case ETHERTYPE_ARP:
            printf("Address resolution\n");
            break;

        case ETHERTYPE_AT:
            printf("Appe Talk protocol\n");
            break;

        case ETHERTYPE_AARP:
            printf("Appe Talk ARRP\n");
            break;
        
        case ETHERTYPE_REVARP:
            printf("Reverse ARP\n");
            break;
        
        case ETHERTYPE_VLAN:
            printf("VLAN tagging\n");
            break;


        case ETHERTYPE_IPX:
            printf("IPX\n");
            break;
       
        case ETHERTYPE_IPV6:
            printf("IPv6\n");
            break;

        case ETHERTYPE_LOOPBACK:
          printf("loop back\n");
          break;
          
        default:
            printf("unknown");
            break;
    }


}

\end{lstlisting}

packetキャプチャにおいてプロミスキャスモードに設定した上で,etherヘッダーを覗き見ることで ターゲット端末のMACアドレスを調べることが可能になる.(etherヘッダは暗号化できないため)



%ここはMacアドレスの内容
\section{Macアドレス ioctlの設定}
ifreq構造体を用いる. ifreq構造体は/linux/if.hで定義されている. 詳しくは man コマンドで確認できる.
一部の内容を次に記す.
%\begin{comment}
\begin{lstlisting}[caption=ifreq, label=ifreq]
struct ifreq 
{
#define IFHWADDRLEN	6
	union
	{
		char	ifrn_name[IFNAMSIZ];		/* if name, e.g. "en0" */
	} ifr_ifrn;
	
	union {
		struct	sockaddr ifru_addr;
		struct	sockaddr ifru_dstaddr;
		struct	sockaddr ifru_broadaddr;
		struct	sockaddr ifru_netmask;
		struct  sockaddr ifru_hwaddr;
		short	ifru_flags;
		int	ifru_ivalue;
		int	ifru_mtu;
		struct  ifmap ifru_map;
		char	ifru_slave[IFNAMSIZ];	/* Just fits the size */
		char	ifru_newname[IFNAMSIZ];
		void *	ifru_data;
		struct	if_settings ifru_settings;
	} ifr_ifru;
};
#define ifr_name	ifr_ifrn.ifrn_name	/* interface name 	*/
#define ifr_hwaddr	ifr_ifru.ifru_hwaddr	/* MAC address 		*/
#define	ifr_addr	ifr_ifru.ifru_addr	  /* address		*/
#define	ifr_dstaddr	ifr_ifru.ifru_dstaddr	/* other end of p-p lnk	*/
#define	ifr_broadaddr	ifr_ifru.ifru_broadaddr	/* broadcast address	*/
#define	ifr_netmask	ifr_ifru.ifru_netmask	 /* interface net mask	*/
#define	ifr_flags	ifr_ifru.ifru_flags	/* flags		*/
#define	ifr_metric	ifr_ifru.ifru_ivalue	/* metric		*/
#define	ifr_mtu		ifr_ifru.ifru_mtu	/* mtu			*/
#define ifr_map		ifr_ifru.ifru_map   	/* device map		*/
#define ifr_slave	ifr_ifru.ifru_slave	/* slave device		*/
#define	ifr_data	ifr_ifru.ifru_data	 	/* for use by interface	*/
#define ifr_ifindex	ifr_ifru.ifru_ivalue	/* interface index	*/
#define ifr_bandwidth	ifr_ifru.ifru_ivalue   /* link bandwidth	*/
#define ifr_qlen	ifr_ifru.ifru_ivalue	/* Queue length 	*/
#define ifr_newname	ifr_ifru.ifru_newname	 /* New name		*/
#define ifr_settings	ifr_ifru.ifru_settings	/* Device/proto settings*/

\end{lstlisting}
%\end{comment}

注目すべきポイントは ifreq構造体に ifr\_hwaddr要素がある, これは 厳密にはifru\_hwaddrとして宣言されているが,
\verb|#|defineの部分で  ifr\_ifru.ifru\_hwaddrにすると書いてある( \verb|#| define ifr\_hwaddr	ifr\_ifru.ifru\_hwaddr の部分.)
よく見ると ifr\_hwaddr構造体は struct  sockaddr ifr\_hwaddr; となっており,これはsockaddr構造体で宣言されている.つまりMACアドレスを見るにはsockaddr構造体としてアクセスする必要がある.
sockaddr構造体は /include/bits/socket.hに次のように定義されている.

\begin{lstlisting}[caption=sockaddr, label=sockaddr]
struct sockaddr
  {
    __SOCKADDR_COMMON (sa_);	/* Common data: address family and length.  */
    char sa_data[14];		/* Address data.  */
  };
\end{lstlisting}
このsa\_data[14]のうち, sa\_data[0] \verb|~| sa\_data[5]の部分にMACアドレスが格納される. sa\_data[6] \verb|~| sa\_data[13]は0が入る.

\subsection{MACアドレスの表示}
インターフェースの操作には ifreq構造体と ioctlシステムコールを用いる.
ioctlシステムコールを用いるには事前に,リンクレイヤを扱うためにSOCK\_RAWを指定したソケットディスクリプタが必要となる.
ifreqメンバの一つであるifr\_nameにはインターフェース名が入り, これはターミナル上で iifconfigコマンドを用いて表示される(en0, eth0など)
続いて iosctlシステムコールを用いる.
SIOCGIFHWADDR フラグを用いてインタフェースのMACアドレスを取得する.このフラグを設定するとifr\_hwaddrメンバにアドレスを格納してくれるようになる.
以下がMACアドレスを表示するプログラム全体である.
\begin{lstlisting}[caption=showMACaddress, label=showMACaddress]
#include<stdio.h>
#include<string.h>
#include<unistd.h>
#include<sys/ioctl.h>
#include<sys/socket.h>
#include<net/if.h>
#include<net/ethernet.h>
#include<netpacket/packet.h>
#include<arpa/inet.h>
 
int main(void)
{
  struct ifreq ifreq;
  int s;
  if( (s=socket(PF_PACKET, SOCK_RAW, htons(ETH_P_ALL)))<0 )
  {
    perror("socket"); //creating a raw socket in advance
    exit(1);
  }
  
  memset(&ifreq, 0, sizeof(struct ifreq));
  strncpy(ifreq.ifr_name, "enp0s3", sizeof(ifreq.ifr_name)-1);
  if(ioctl(s, SIOCGIFHWADDR, &ifreq)!=0) perror("ioctl");
 
  printf("%02x:%02x:%02x:%02x:%02x:%02x \n",
  (unsigned char)ifreq.ifr_hwaddr.sa_data[0],
  (unsigned char)ifreq.ifr_hwaddr.sa_data[1],
   (unsigned char)ifreq.ifr_hwaddr.sa_data[2],
   (unsigned char)ifreq.ifr_hwaddr.sa_data[3], 
   (unsigned char)ifreq.ifr_hwaddr.sa_data[4], 
   (unsigned char)ifreq.ifr_hwaddr.sa_data[5]);
  
  close(s);
  return 0;
}
\end{lstlisting}

\subsection{MACアドレスの偽装}
MACアドレスはデバイス固有の番号であり,NICカードに記載されており,書き換えはできない.
しかし,プログラムにより, OS側に対して偽のMACアドレスを認識できるように書き換えることはできる.
このことからMACアドレスだけを用いるのはセキュリティ面では不十分.
このプログラムを試すのは仮想環境で行うことを勧める.
MACアドレスの書き換えは ifreq.ifr\_hwaddr.sa\_data[15]配列を書き換えた後に
ioctlのフラグを SIOCSIFHWADDRにするだけである.
なお,MACアドレスは16進数の6桁からなるが,実際に書き換えることができるのは下位3桁であり,上位3桁はOUI(ベンダーコード)と呼ばれ,デバイスを作った製造元の値が入るため, この部分の書き換えになると ioctlでエラーが発生する.
(もしかしたらうまく書き換える方法があるのかもしれない)
以下がソースコードである.

\begin{lstlisting}[caption=changeMACaddress, label=changeMACaddress]
#include<stdio.h>
#include<string.h>
#include<sys/ioctl.h>
#include<sys/socket.h>
#include<net/if.h>
#include<net/ethernet.h>
#include<arpa/inet.h>
 
int main(void)
{
  struct ifreq ifreq;
 
  int s;
  if( (s=socket(PF_PACKET, SOCK_RAW, htons(ETH_P_ALL)))<0) 
  {//creating a raw socket
    perror("[-]socket()");
    exit(1);
  }
  
  memset(&ifreq, 0, sizeof(struct ifreq));
  strncpy(ifreq.ifr_name, "eth1", sizeof(ifreq.ifr_name)-1);
  if(ioctl(s, SIOCGIFHWADDR, &ifreq)<0) perror("[-]ioctl()");
 
  printf("Mac Address\n");
  printf("before <%02x:%02x:%02x:%02x:%02x:%02x>\n",
	 (unsigned char)ifreq.ifr_hwaddr.sa_data[0],
	 (unsigned char)ifreq.ifr_hwaddr.sa_data[1],
	 (unsigned char)ifreq.ifr_hwaddr.sa_data[2],
	 (unsigned char)ifreq.ifr_hwaddr.sa_data[3],
	 (unsigned char)ifreq.ifr_hwaddr.sa_data[4],
	 (unsigned char)ifreq.ifr_hwaddr.sa_data[5]);
 
  
  ifreq.ifr_hwaddr.sa_data[0] = 0x08;//OUI
  ifreq.ifr_hwaddr.sa_data[1] = 0x00;//OUI
  ifreq.ifr_hwaddr.sa_data[2] = 0x27;//OUI
  ifreq.ifr_hwaddr.sa_data[3] = 0xaa;
  ifreq.ifr_hwaddr.sa_data[4] = 0xaa;
  ifreq.ifr_hwaddr.sa_data[5] = 0xaa;
  if(ioctl(s, SIOCSIFHWADDR, &ifreq)<0)perror("[-]ioctl()");
 
  printf("after <%02x:%02x:%02x:%02x:%02x:%02x>\n",
	 (unsigned char)ifreq.ifr_hwaddr.sa_data[0],
	 (unsigned char)ifreq.ifr_hwaddr.sa_data[1],
	 (unsigned char)ifreq.ifr_hwaddr.sa_data[2],
	 (unsigned char)ifreq.ifr_hwaddr.sa_data[3],
	 (unsigned char)ifreq.ifr_hwaddr.sa_data[4],
	 (unsigned char)ifreq.ifr_hwaddr.sa_data[5]);
 
  return 0;
}

\end{lstlisting}

\section{IPアドレス ioctlの設定}
ioctlシステムコールを用いる,フラグに次の二つを用いる.
SIOCGIFADDRで指定したインターフェスのIPアドレスを取得し,その情報をifreq構造体のifr\_addrメンバに格納する.
SIOCSIFADDRで指定したインターフェスのIPアドレスを取得し,その情報をifreq構造体のifr\_addrメンバに書き込む.
ioctlを用いるためには RAWのソケットディスクリプタをあらかじめ生成する必要がある.

\begin{lstlisting}[caption=changeIPaddress, label=changeIPaddress]
#include<stdio.h>
#include<sys/ioctl.h>
#include<sys/types.h>
#include<sys/socket.h>
#include<unistd.h>
#include<string.h>
#include<arpa/inet.h>
#include<net/if.h>
 
int main(void){
  struct ifreq ifreq;
  struct sockaddr_in *sin;
  char buf[INET_ADDRSTRLEN];
 
  int s;
  if( ( s=socket(PF_PACKET, SOCK_RAW, 0) )<0 ){
    perror("[-]socket()");
    exit(1);
  }
  
  //get IP address
  memset(&ifreq, 0, sizeof(struct ifreq));
  strncpy(ifreq.ifr_name, "eth1", sizeof(ifreq.ifr_name)-1);
  if(ioctl(s, SIOCGIFADDR, &ifreq)<0){
    perror("[-]ioctl(SIOCGIFADDR)");
    close(soc);
    return 0;
  }
 

  sin = (struct sockaddr_in *)&ifreq.ifr_addr;
  inet_ntop(AF_INET, &sin->sin_addr.s_addr, buf, sizeof(buf));
  printf("[+]Before: %s\n", buf);
 
  // write out IP address 
  sin->sin_addr.s_addr = inet_addr("10.24.94.100");
  if(ioctl(s, SIOCSIFADDR, &ifreq)<0){
    perror("[-]ioctl(SIOCSIFADDR)");
    close(s);
    return 0;
  }
  else printf("[+]changed IP address\n");
 
  close(s);
  return 0;
}
\end{lstlisting}

IPアドレスの変更だけでなく,
サブネットマスクとブロードキャストアドレスを変更することも行う必要がある,
ioctlのフラグとして次の二つを用いる.
SIOCSIFBRDADDR これはifreq構造体のifr\_broadaddrメンバに格納された情報を指定したインターフェスのブロードキャストアドレスとして書き込む(更新する)
SIOCSIFNETMASK これはifreq構造体のifr\_netmaskメンバに格納された情報を指定したインターフェスのサブネットマスクとして書き込む(更新する)
コードを追加.
\begin{lstlisting}[caption=changeSubnetmaskandBroadcastaddress, label=changeSubnetmaskandBroadcastaddress]
sin = (struct sockaddr_in *)&ifreq.ifr_broadaddr;
  sin->sin_addr.s_addr = inet_addr("10.24.95.255");
  if(ioctl(soc, SIOCSIFBRDADDR, &ifreq)<0){
    perror("[-]ioctl(SIOCSIFBRDADDR)");
    close(soc);
    return 0;
  }else printf("[+]set broadcast address\n");
 
  sin = (struct sockaddr_in *)&ifreq.ifr_netmask;
  sin->sin_addr.s_addr = inet_addr("255.255.240.0");
  if(ioctl(soc, SIOCSIFNETMASK, &ifreq)<0){
    perror("[-]iocrl(SIOCSIFNETMASK)");
    close(soc);
    return 0;
  }else printf("[+]set network mask\n");

\end{lstlisting}

\section{ファイル処理}
openシステムコール,stat,fstatシステムコールを用いる.
stat関数は次のような書式になっている.

int stat(const char *path, struct stat *buffer)

ファイルのありかをpathに格納し, stat構造体の変数 buffに格納する.

一方でfstat関数の書式は次のようになっている.

int fstat(int fd, struct stat *buffer)

fdはファイルディスクリプタである.
ファイルディスクリプタfdを用いて

fd=open(path,O\_RDONLY,0) 

でファイルを開くことができる.
両者ともに構造体stat のインスタンスを作り, stat関数に参照と,ファイルディスクリプタを渡す.

stat構造体は以下のようになっている. 詳しくはmanコマンドで参照できる.
%\begin{comment}
\begin{lstlisting}[caption=structStat, label=structStat]
struct stat {
    dev_t     st_dev;     /* device ID */
    ino_t     st_ino;     /* inode num */
    mode_t    st_mode;    /* access */
    nlink_t   st_nlink;   /* the number of hard link */
    uid_t     st_uid;     /* user  ID */
    gid_t     st_gid;     /* user group ID */
    dev_t     st_rdev;    /* device ID  */
    off_t     st_size;    /* total size (byte unit) */
    blksize_t st_blksize; /* file I/O block size */
    blkcnt_t  st_blocks;  /* alloacted 512B block num */
};
 \end{lstlisting}
%\end{comment}


このstat構造体のメンバの一つである st\_sizeにはファイルの大きさを格納されるため,ファイルの大きさ取得にはこれを用いる.
以下のコードはpath="/Users/fujiwara/desktop/make-practice/makeserver/file/test.txt";
に存在するtest.txtのファイルの大きさを取り出す関数である.
fstatには ファイルディスクリプタfd , statにはファイルのありかのpathを指定している.
ともにstat構造体の参照を引数に取り入れていることに注目する.
\begin{lstlisting}[caption=fileexample, label=fiileexample]
#include<stdio.h>
#include<fcntl.h>
#include<stdlib.h>
#include<sys/stat.h>

int main(){
    struct stat statBuf;

    const char path[]
    	="/Users/fujiwara/desktop/make-practice/makeserver/file/test.txt";
    
    int size=0;
    if(stat(path,&statBuf)==0){
        size = statBuf.st_size;
    }

    fprintf(stdout,"%d\n",size);
    
    int fd;
    if( (fd = open(path,O_RDONLY,0) )<0){
        fprintf(stdout,"Couldn't open file");
        exit(1);
    }   

    if(fstat(fd,&statBuf)==-1){
        fprintf(stdout,"Couldn't open file");
        exit(1);
    }
    size = statBuf.st_size;
    
    fprintf(stdout,"%d\n",size);
    
    return 0;

}
\end{lstlisting}

このように socket()関数と同じような手順でファイルを扱うことができる点に注目してほしい.


\end{document}



